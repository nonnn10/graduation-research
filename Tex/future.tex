\chapter{結論}

\section{まとめ}
本研究では船舶の運行状況が当日にしか公表されないという問題を機械学習を用い、比較的安易に入手可能な気象状況の値や気象予報図を適用し数日先までを予測する実験を実施した。画像、数値データどちらも9日先までの予測実験を行い、数値用いた予測では特徴量と教師ラベルとの関係を明らかにすることで、学習モデルが重要視している特徴を確認することができた。特徴量の分散をみることで運航データと欠航データとの特徴量の相関を確認し、予測を行うことにおいて重要な特徴量を確認することができた。画像を用いた予測では航路ごとの入力画像を確認することで、どのような画像がモデルに対して影響を与えているかを示した。また、画像分類では鳩間島、西表上原航路において当日から7日前モデルまでは評価指標が減少する傾向にあったが8日前モデルから上昇することを発見した。また、これらの考察から数値、画像データを時系列的に扱うことで特徴の変遷を保持することができると考えられる。

\section{今後の課題}
数値と画像どちらのデータも特徴があり、一方だけで無く両方のデータを活用することでデータの特徴生かしたモデル、データセットを構築することで予測精度向上が期待できるのではないかと考えた。